\subsubsection{Scope}

The Telemetry Gateway is a service that recieves information from
trusted sources outside the Critical Operations Enclave and injects
that information as telemetry into the OCS System.

\subsubsection{Overview}

\paragraph{Description}

The Telemetry Gateway is a service that recieves information from
trusted sources outside the Critical Operations Enclave and injects
that information as telemetry into the OCS System. 

\paragraph{Objective}

The prime use cases supported by the Telemetry Gateway are:

\begin{itemize}
\item The ingest of image parameters (WCS, PSF, or similar) from both prompt
processing and offline Level 1 processing use cases.
\item Return of data from OCS-submitted batch jobs to the OCS System.
\end{itemize}

\paragraph{Operational Context}

The Telemetry Gatewaying Service is an OCS-commandable device which runs
under the control of Observatory Operations.

The service transmits information originating from outside the Critical Operations 
Enclave to inside that enclave.

\subsubsection{Operational Concepts}

\paragraph{Normal Operations}

\subparagraph{Operations During Prompt Proccessing}

In normal operations, the service runs while prompt processing is running,
or when OCS-driven batch jobs requesting return of parameters are running.  
The orchestration responsibility arranging for the telemetry gateway belongs to OCS.

\subparagraph{Operations Supporting L1 Offline Processing}
The service also runs at appropriate times to ingest delayed telemetry
generated by offline Level 1  processing. There is no need for the service
to run concurrently with Level 1 offline processing, as offline processing is
generated in a batch system, and the telemetry gateway runs according to
the availablilty of the OCS system. This implies outputs from offline L1 processing
are buffered for ingest via the telemetry gatewaying service.

\paragraph{Operational Scenarios}

\subparagraph{Change Control}

Upgrades to the Telemetry Gatewaying Service are produced in the LSST Data Facility. 
Change control of this function is coordiniated with the Observatory, with the 
Observatory having an absolute say about insertion and evaluation of changes.