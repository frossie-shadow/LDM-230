\subsubsection{Scope}

This section describes the concept of operations for archiving raw data acquired 
from instruments on the Auxiliary Telescope to the permanent archive.

\subsubsection{Overview}

\paragraph{Description}

The Auxiliary Telescope is a separate telescope at the Summit site, located on a 
ridge adjacent to that of the main telescope building. This telescope supports a 
spectrophotometer that measures the light from stars in very narrow bandwidths 
compared to the filter pass bands on the main LSST camera. The purpose of the 
spectrophotometer is to measure the absorption, which is how light from astronomical 
objects is attenuated as it passes through the atmosphere. By pointing this instrument 
at known “standard stars” that emit light with a known spectral distribution, it is 
possible to estimate the extinction. This information is used to derive photometric 
calibrations for the data taken by the main telescope. 

The Auxiliary Telescope camera produces 2-dimensional CCD images, but the headers 
and associated metadata are different than the LSSTCam data because spectra, not 
images of the sky, are recorded. The Auxiliary Telescope slews 1:1 with the main LSSTCam, 
which implies two exposures every ~39 seconds.

From the point of view of LSST Data Facility Services for Observatory Operations, the 
spectrograph on the Auxiliary Telescope is an independent instrument that is controlled 
independently from the main LSSTCam. Thus, the operations of and changes to LSST 
Data Facility services for this instrument must be decoupled from all others.

The Auxiliary Telescope and its spectrograph are devices under the
control of the observatory control system (OCS). The spectrograph
contains a single LSST CCD. The Camera Data System (CDS) for the single CCD in the 
spectrograph uses a readout system based on the LSSTCam electronics and will
present an interface for the Archiver to build FITS files. Telescope data products 
are described \citeds{LSE-140}.

\paragraph{Objective}

The Spectrograph Archiving Service reads pixel data from the Spectrograph verison of 
the CDS and metadata available in the overall Observatory Control System and builds FITS files. 
The service archives the data in a way that the data are promptly available to Observatory 
Operations via the Observatory Operations Data Service, and that the data appear in the Data 
Backbone.

\subsubsection{Operational Concepts}

Archiving is under control of OCS, with the same basic operational
considerations as the CCD data from LSSTCam. Keeping
in mind the differences between the two systems, the concept of
operations for LSSTCam archiving apply (see section on LSSTCam Archiving Service). 
One differing aspect is that these data are best organized temporally, 
while some data from LSSTCam are organized spatially.

There is no prompt processing of Spectrograph data in a way that is
analogous to the prompt proceessing of LSSTCam data.

\paragraph{Normal Operations}

Under normal operations the Spectrograph Archiving Service is under control of the Observatory
Control System.

\paragraph{Operational Scenarios}

\subparagraph{Change Control}
Upgrades to the Spectrograph Archiving Service are produced in the LSST Data Facility. 
Change control of this function is coordiniated with the Observatory, with the Observatory 
having an absolute say about insertion and evaluation of changes.  