\subsubsection{Overview}
This section briefly describes functions and processes of service management that are implemented across all service and ITC layers of the LSST Data Facility. These elements were drawn from the Information Technology Infrastructure Library (ITIL) which is an industry-standard vocabulary for IT service management.

IT Service Management processes include

\begin{enumerate}

\item Service Design: Building a service catalog and arranging for changes to the service offering, including internal supporting services.

\item Service Transition: Specifying needed changes, assessing the quality of proposed changes,
and controlling the order and timing of inserting changes into the system.

  \begin{itemize}
 
  \item \emph{Change Management} provides authorization for streams of changes to be requested, for the insertion of changes into the reliable production system, and for the assessment of the success of these changes.

  \item \emph{Release Management} interacts with project producing a specific change to ensure that
a complete change is presented to change management for approval into the live system. Examples areas that are typically a concern are accompanying documentation and security aspects.

  \item \emph{Configuration Management} provides an accurate model of the components in the live system sufficient to understand changes, and support operations.
 
  \end{itemize}

\item Service Delivery: operating the current set and configuration of production services. Service delivery processes must satisfy the detailed service delivery concepts presented elsewhere in this document.

  \begin{itemize}
  
  \item \emph{Request Response}
  
  \item \emph{Incident Response}
  
  \item \emph{Problem Management}
  
  \end{itemize}
  
\end{enumerate}