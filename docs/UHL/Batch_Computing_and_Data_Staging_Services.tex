\subsubsection{Scope}

Batch Computing and Data Staging Services provide primitives 
used by the Master Batch Job Scheduling Service. Batch Computing 
and Data Staging Services are provides in a distinct implementation 
for that is tailored for each batch system deployment.

\subsubsection{Overview}

Batch Computing and Data Staging Services are provided at NCSA and the Base Center.

Analogous (but not identical) services are provided by MoU to the LSST
Data Facility by CC-IN2P3, as well as by any commercial batch
provisioning and agency resources, such as XSEDE. 

\paragraph{Description}

Both NCSA and the Base Center will have a core batch infrastructure that
uses batch system logic to partition a pool of batch resources to
various enclaves at the respective sites, with policies that govern
priorities and file systems exposed for batch nodes running in the
context of each enclave.

At the Base Center, Batch Services are supplied to the Commissioning
Cluster and the Chilean Data Access Center from this pool.

At NCSA, Batch Services are supplied to the Development, Integration,
General Production, L1 and US Data Access Centers enclaves from this pool.

An additional pool of batch resources at each site is drawn from idle nodes
in the core Kubernetes provisioning. An enhanced goal is the unification of
resource management of the Kubernetes nodes and the batch pool.

Data staging refers to mechanisms needed to move data, primarily files, between the
Data Backbone and the storage used by batch programs. This may be as simple
as a copy operation between mounted file systems, or as complex as a staging
via http or FTP.

\paragraph{Objective}

Batch Computing and Data Staging Services support LSST batch 
operations by providing a batch system supported by data movement primitives. 

\begin{itemize}

\item Provide a batch scheduler.

\item Provide any enclave-specific resources. An example is distinct head
nodes for different enclaves.

\item Provide enclave-specific configurations, including configurations
needed for information security and work processes.

\item Integrate ITC into the batch system.

\end{itemize}

\paragraph{Operational Context}

Batch Computing and Data Staging Services use resources in the master provisioning 
enclave and expose them to a given enclave, implementing policies appropriate to that enclave.

\subsubsection{Operational Concepts}

\paragraph{Operational Scenarios}

An important consideration is that these resources do not have a constant
level of use within each enclave, and that over time the hardware resources
needed for batch operations in an enclave will change.

Operating conditions may change as well. For example even with container
abstractions, it may be necessary to partition the batch resources to
support two versions of an operating system.

Somewhat analogously, the batch system may opportunistically use idle nodes provisioned
for elastic Kubernetes computing. 

Lastly, NCSA has substantial resources for prompt processing, such as alert
production. Scheduling jitter and performance may preclude using a single batch
scheduler for general offline production and prompt processing. Batch Computing 
and Data Staging Services covers having multiple scheduler instances.