\subsubsection{Scope}

This section describes network-based operational information security services
supporting the Observatory Operations and the LSST Data Facility.

\subsubsection{Overview}

\paragraph{Description}

The LSST Network-based IT Security Service provides technical controls for
operational security assurance. These controls provide data that support the
LSST Master Information Security Plan and IT security processes such as incident
detection and resolution.

\paragraph{Objective}

The objective of the Network-based IT Security Service is to provide:

\begin{itemize}
\item Network Security Monitoring, including monitoring of high-rate data
connections for data transfer across the LDF system boundaries (but excluding
certain high rate transfers, such as the Level 1 service access to the Camera
Data System), including deployment of technologies for Active Response and
Blocking of Attacks.
\item Vulnerability Management for computers and application software in the
enclaves.
\item The technical framework to facilitate efficient Incident Detection and
Response, including central log collection/event correlation for security purposes.
\item Management of certain access controls, such as firewalls and bastion hosts,
used for administrative access.
\item Host-based intrusion detection clients deployed on end systems as
appropriate.
\item Security configuration management and auditing to baseline standards.
\end{itemize}

\paragraph{Operational Context}

The general approach to operational information security is that there is a LSST
Information Security Officer (ISO) who reports to the Head of the LSST
construction project, and will transition to report to the Head of the Operations
project.

The ISO drafts a Master Security Program (citeds{LPM-121}) plan, which the Head
approves of as appropriately mitigating the Information Security risk. The Head
then assumes responsibility for the residual risk of the plan. This is the
security risk that remains, given faithful execution of the plan. The ISO
oversees implementation and evolution of the plan, seeing that it is faithfully
implemented and noting when mitigation and changes are needed. The ISO does any
required staff work for the Head; for example, running staff training. The ISO
is informed of and keeps records on security incidents, and is responsible for
evolution of the security plan and evolution of security threats.

The ISO is responsible for a Information Security Response Team, which deals
with actual or latent potential breaches in information security. The Incident
Response Team is made of a set of draftees from the various operations
departments, with the draft weighted towards departments with expertise and
responsibility for critical operations and critical information security needs.

 The ISO runs the annual security plan assessment. The management of each
 construction subsystem and operations department is responsible for annual
 revision of a Departmental Security Plan that complies with the Master Plan.
 These departmental plans include

\begin{itemize}
\item A comprehensive list of IT assets, applications and services.
\item A list of security controls the department applies to each asset
(technical and operational).
\item A list of controls supplied by others that are relied on.
\end{itemize}

These controls apply to all offered services and all supported ITC. Reporting
is easiest if the systems offered are under good configuration control. Under a
good system, the security plans are living and updated by an effective change
control process.

Verification: The ISO oversees a group that provides network-based security
services described in the Objective part of this concept of operations.

A general approach to LSST-specific networking is the use of software-defined
networking. This provides for isolation of networking supporting security enclaves.
In particular, this allows for the separation of critical infrastructure for
Observatory Operations and the LSST Data Facility from office or other routine
networking.

The context for these security enclaves cover the following production services
in the LSST project, though other enclaves may join if feasible and desired by
the relevant operational partners. These networks may participate in this
infrastructure, but are currently seen as the responsibilities of AURA and NCSA.

\begin{longtable}{|p{0.3\textwidth}|p{0.5\textwidth}|} \hline
\textbf{Production Service} & \textbf{Security Enclave} \\\hline
Level One Services & Split Between NCSA and Chile  \\\hline
Batch Production services & NCSA portion, excluding satellite center \\\hline
US Data Access Center Services & NCSA \\\hline
Critical Observing Enclave Services & Summit and Base Center \\\hline
Chilean Data Access Center Services & Base Site \\\hline
Data Backbone Services & Base Center and NCSA \\\hline
\end{longtable}

\paragraph{Risks}

These are the standard elements of an information security infrastructure which
are needed for a credible IT security project. Certain elements of the system
are near the state of the art due to the data rates involved. Lack of credible
infrastructure in this area will be seen as a flaw in the overall construction
plan, preparing the LSST MREFC for operations.

\subsubsection{Operational Concepts}

\paragraph{Normal Operations}

The following elements provide the functionality needed to implement the
network-based security elments of the LSST Master Information Security Plan:

\begin{itemize}

\item Intrusion Detection Systems (IDS) detect patterns of network activity that
indicate attacks on systems, compromise of systems, violations of Acceptable Use
of systems, abuse of systems, and other security-related matters.

\item Vulnerability Scanning detects software services with vulnerable
configurations or unpatched versions of software via network fingerprinting. The
system scans designated systems subject to a black-list. In addition to scanning
for vulnerabilities, port scanning for firewall audits and ARP scanning for
network asset management can also be conducted.

\item Central Log Collection and Event Generation collects syslogs and other
designated logs for storage (making logs invulnerable for an attacker to modify)
and processes the logs to detect signatures indicating a compromise or poor
security practices.

\item Firewalls and bastion hosts provide a layer of active security. A typical
use of a bastion host is to provide a layer of security between networks used to
administer computers and more general networks.

\item Host-based Intrusion Detection complements network monitoring by detecting
actions within a system not visible from the network with tools such as auditd
and OSSEC. This component also monitors the filesystems and checks for filesystem
integrity.

\item Active Response blocks communication with entities outside the observation site
networks. This component is typically used to block “bad actor” entities outside
the observation site network.

\item Central Configuration Management enforces a baseline security and configuration
on all systems.

\end{itemize}

New systems being deployed must be ``hardened'' to a security baseline and
vetted by security professionals before moving into operations or after major
configuration changes.

\paragraph{Operational Scenarios}

Vulnerability scanning periodically assesses designated ports on designated
computers sensing vulnerabilities. An example of when this service is applied in
a crucial case is assessing the effectiveness of the program of work patching a
critical vulnerability.

Intrusion detection can detect, for example, an attempt to compromise a system.
The detection system interacts with the active response system to cut off the
attacker’s access to the computer. The intrusion detection system can also be
used to aid in the investigation of an attack during the incident response and
handling of a security incident.

Host-based Intrusion Detection checks for attacks against a host from the
perspective of the host. Examples include multiple failed remote logins as reported
by the host, or reports of file system changes that do not accompany an approved
request for change or do not fall within a maintenance window.

The networks at the Observatory site must be monitored by intrusion detection
systems. Acceptable IDS solutions include Bro and Snort. These systems must be
able to handle the traffic load from various network segments at 10GB to 100GB
speeds. The IDS systems must be placed at strategic locations and account for
any expansion or changes in the network without the need to completely retool
the IDS systems.

The information produced by the system is accessible by LSST staff involved in
LSST information security, “landlords” hosting systems, and other parties with a
valid interest in the data, to the extent required by the site’s specific security plans.

Active Response is typically referred to as a Black Hole Router (BHR) since it peers
with the border routers on a network and offers the shortest router to
destinations being blocked.  Quagga and ExaBGP are two examples of BHR software.

The central Configuration Management System will enforce a security baseline and
configuration on all systems. Examples of this technology are Puppet and Chef.
In the event that Windows systems will be deployed onsite, a system enforcing
Group Policy Objects and WSUS for patching will have to be available. It is
required that this system would also be the Domain Controller using Active
Directory and federate with LSST’s Unified Identity Management system.

The central log collectors are responsible for collecting and archiving all logs
collected as described in the previous section. The collectors must be able to
store at minimum six months of logs with a rotating windows deleting the oldest
logs to maintain disk space. In addition to the log collectors, there is a SEIM/analysis
system. This system is used for real-time log alerts, searches, and
visualization. ElasticSearch, Kibana, and OSSEC are three examples of such
software. This server spools a copy of the logs from the central collectors but
may not be able to keep the full time window due to overhead or log metadata
storage.

All systems, both workstations and servers, are required to send system logs to
a central collector. For Linux systems, syslog must be configured to send a copy
of all logs in realtime to the central collector. For Windows systems, software
such as Snare will be installed to send Windows Event logs to the central
collector using the syslog protocol. Other alternatives exist for log collection
such as logstash an open source log collection tool that can be used to collect
logs from a wide variety of platforms.

Network devices are also required to send system logs to the central collector.
Note that this is different than any network logs a switch or router would send.
The system logs refer to events such as device logins, configuration changes, and
other system specific events. Note that this is a requirement only if the device
has this feature available.

Network devices such as routers or firewalls that are placed on the ingress/egress
points of a vlan or the network must send firewall or router ACL logs to the
central collector.

It is best practice for network devices to also send netflow to a central
collector. However, if netflow is collected and forwarded to the central
collector it must not be at the expense of network or device performance.

Any other devices not classified as a server, workstation, or networking device
must be configured to send logs to the central collector if this feature is
available. An example of a device that falls into this category is a VOIP appliance
 or a VPN appliance.
